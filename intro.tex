\section{Introduction}



Throughout the rest of the paper, we will use the following glossary when describing the components of the neural and relating them to a QEC workflow:
\begin{itemize}
\item Round: A single, complete set of stabilizer measurements over the circuit that only leaves logical observables as additional degrees of freedom.
\item Cycle: A set of stabilizer measurement rounds upon which either error correction is applied (or errors are tracked) or the data qubits are measured. We will reserve the symbol $r$ to refer to the total number of rounds in a cycle.
\item Kernel distance, $k$: The distance of the underlying surface code that is used in the convolution kernels of the NN architecture. The kernel construct is described in Section~\ref{sec:kernels}, and we will use the value $k=3$ for most of the discussion in this paper.
\item Surface code with distance $d$: When referring to surface codes, we will explicitly refer to rotated surface codes~\cite{Bombin:2007}. While the implementation of our NN architecture assumes this configuration, future investigations could easily account for unrotated surface codes or other structured codes, \eg, the more general class of low-density parity-check codes, using similar concepts.
\item $z$-like variable: A real-valued NN parameter or derived variable that can take values within $(-\infty, \infty)$. For most of the $x$-like, or $p$- and $f$-like variables, described below, that are obtained from an underlying $z$-like value, this $z$-like value is clipped between $\pm12$ in order to avoid singularities in derivatives, which arise fundamentally from the floating-point precision issue $e^z \to 0$ (1) for very small (large) signed values of $z$.
\item $x$-like variable: A real-valued transformed NN variable that can take values within $[0, \infty)$. As done in this paper, one can obtain such values using the transformation $x(z)=e^z$, with the useful property that $x(-z)=1/x(z)$.
\item Sigmoid activation/transformation: There are many options in the literature to transform a $z$-like variable into values bounded asymptotically as $z \to \pm \infty$. When we use this term in this paper, we will explicitly refer to the transformation $\sigma(z)=\left(1+e^{-z}\right)^{-1}$, with the useful property that $\sigma(-z)=1-\sigma(z)$.
\item $p$- (probability) or $f$- (fraction) like variable: A real-valued NN variable that can take values within $(0, 1)$, or $[0,1]$ if the bounds are included explicitly in an embedding parametrization. In the case of exclusive bounds, these variables are obtained using a sigmoid transformation.
\item $c_\varphi$- or $\alpha$-like parameters: A real-valued NN variable that can take values within $(-1, 1)$. These variables can be obtained via a hyperbolic tangent transformation over $z$-like trainable variables. While the way to obtain these parameters are the same, we will distinguish their meaning as the cosine of a phase difference for $c_\varphi$, and a $z$-like variable sign inversion control parameter for $\alpha$.
\item Detectors/detector events: A set of measurements that can be used to flag potential errors. The formalism to define detector events is explained in Refs.~\cite{Gidney:2021,Higgott:2023,Bausch:2023jgi}. One example for detectors would be an XOR of the readouts of the same measure qubits over two consecutive rounds. While stabilizer measurement values are denoted with the reserved letter $m$, detector events are denoted with the reserved letter $e$ throughout this paper, unless it is clear from the context that we are referring to exponentiation using Euler's number.
\end{itemize}
